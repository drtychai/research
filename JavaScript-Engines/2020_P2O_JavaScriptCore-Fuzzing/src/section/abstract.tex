% Abstract
\renewcommand{\abstractname}{}
%\twocolumn[\begin{@twocolumnfalse} % COMMENT THIS OUT TO ENABLE SINGLE COLUMN MODE
    \maketitle
    \begin{abstract}
        \vspace{-10ex}
        \centering
        \begin{minipage}{\dimexpr\paperwidth-5cm}
          %------------------------------------------------------------------------
          \ab{
              \textsc{\bf{Abstract.\hspace{1ex}}}%
            \textrm{\noindent Modern browser exploitation has shifted the focus of bug hunting to the browser's JavaScript engine. This component often 
              trends towards high cyclomatic complexity and provides the greatest level of flexibility to a vulnerability researcher. This
              has lead to a variety of bugs in varying subcomponents;  from compiler-, interpreter-, and optimization-based vulnerabilities
              to those triggered from garbage collection, WebAssembly, or the DOM.%
              %Are these vulnerabilities trivial to discover? Are there no more left to discover?%
              %
              We present here, an approach to vulnerability research in modern JavaScript engines. We will primarily focus on \textit{Safari}'s
              internal JavaScript engine, JavaScriptCore. A strong focus on bug discovery, crash identification, and logical reasoning
              provides success and failure over a variety of research approaches, both theoretical and pragmatic. We then outline our
              interpretation on the ideal approach to future work in this relam.%
              %
              %However, the Renderer Process is naturally sandboxed by macOS; the above only makes up Stage 1. Research into general sandbox
              %escapes go back over 20 years, with the first notable exploits performing escapes from the first Java sandbox. Modern
              %research into the Apple Safari sandbox provide detailed and informative resources for us to draw and expand
              %from.
              %
              %A likely path here is Apple's rather bug-full IPC mechanisms, of which there is one major one: Mach ports. They originally were
              %used in the Mach kernel as a way to send messages back and forth from the kernel to various modules, but as XNU, Darwin, and
              %computing as a whole became more complicated, Mach ports became a full back bone for the operating system.
              %However, in their original incarnation, they had no concept of a malicious actor inside the system. Thus, ports run in privileged
              %space, with a security model that has had to be bolted onto them post facto, leading to embarrassing mistakes and a number of
              %vulnerabilities. Other possibilities include macOS's syscall interface, which provides 4 separate styles to perform syscalls
              %in a way that Apple actively discourages developers from using and looking at.
              %%
              %This combination of a large code base (Webkit has ~5 million LOC), a continuously changing modular architecture, and thorough
              %detailed analyses from renowned researchers provides us a fruitful target to develop and hone our research methodologies and skills
              %in the Darwin environment.
            } 
          }
          \keywords{Keywords: javascript, fuzzing, intermediate representation, speculative reasoning}%
          %------------------------------------------------------------------------
          % Journal Edition 
          % \DOI{\urlstyle{same} \url{https:/doi.org/xxxxxxxx}    \hspace{8.5cm}    ISSN 1850-1168 (online)}
          %------------------------------------------------------------------------
        \end{minipage}
        \vspace{7ex}
    \end{abstract}
%\end{@twocolumnfalse}]  % COMMENT THIS OUT TO ENABLE SINGLE COLUMN MODE
