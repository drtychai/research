%===============================================================
\section{Future Work}
% iterate off of Fuzzilli's initial work
% fuzzing using AOT compilation of the engine's IR (feed directly/bypass parser)

\subsection{Lessons Learned}
Thus far, we have demonstrated that effective fuzzing leverages a multitude of techniques 
to discover bugs, many of which require a great deal of education and practice. A recent emerging one utilizes
\textit{semantic-aware fuzzing} to drive both the generative and mutation engines.
%
Skyfire\cite{skyfire_2017} was one of the earliest research efforts to tackle the semantic problem in language 
fuzzing; it learns the semantics of a language from existing test cases, in the form of probabilistic 
context-sensitive grammar (PCSG). This grammar is then used for further fuzzing. Conversely, with a focus on 
stressing specific components in a JavaScript Engine, DIE\cite{die_2020} analyzes and utilizes the overall 
semantic properties of each existing test case (i.e., aspects).
%
% Due to the high severity of vulnerabilities in JavaScript runtime environments, a number of research work have attempted
% to fuzz JavaScript engines for finding vulnerabilities [24, 27, 29, 34, 37, 40, 41, 55, 56]. For fuzzing JavaScript
% engines, most research work have focused on generating syntactically valid JavaScript test cases [24, 29, 37, 41, 55].
% Despite their successful fuzzing efforts, they did not consider JavaScript semantics, and thus, could not generate test cases
% effectively—many test cases merely end up as JavaScript runtime errors [27, 56]. Skyfire [56] was proposed to generate
% test cases through the probabilistic context-sensitive grammar that defines syntax features and semantic rules by learning
% from existing samples. CodeAlchemist [27] was proposed to generate semantically-aware JavaScript code by using small
% code blocks collected from a large corpus. Unfortunately, there is no JavaScript engine fuzzer that can generate semantically
% correct test cases all the time.
% 
% Recently proposed JavaScript engine fuzzers tried to reduce the input space. DIE [40] has proposed two mutation strategies,
% structure-preserving mutation and type-preserving mutation. They, also, reduce the input space by utilizing known proof
% of concept (PoC) exploits or existing test cases. Montage [34] showed its outperformed efficacy by leveraging a neural net-
% work language model (NNLM) to generate test cases based on code fragments of previously reported vulnerabilities, similar
% to DIE. Their (DIE and Montage) design choice allowed them to overcome the fundamental limitation of other JavaScript
% engine fuzzers: simply producing generic test cases is not really effective to find vulnerabilities in JavaScript engines
% because of the huge search space.
%
In this section, we propose a future (alternative) direction for continued reserach of JavaScript engines. Unsurprisingly, we, too,
start by tackingly the semantic problem of language -- specifically, ECMAScript.


%---------------------------------------------------------------
%%%
%% something something << aggregate type >> 
%%%
\subsection{A New Approach to the Semantic Problem}
ECMA-262\cite{ecma-2021}, the official language specification for which JavaScript is derived, is over 800 pages. Compared to the
current ISO International C++20 standard\cite{cpp20} (~1,800 pages), less than a thousand pages should be \textit{more}
than approachable -- right? (hell no) So why does JavaScript feel so much larger of a language... 

\begin{center}
\begin{minipage}{0.75\dimexpr\paperwidth}
\p{\hspace{-1.5em}\textit{An Aside...\\}}
\hspace{-1em}JavaScript is a prototype-based, weakly-typed, dynamic language. For 
those defined, a \textit{type object} is a JavaScript object representing a type. Type objects define 
the layout, stride, and size of a continuous region of memory. There are three basic 
categories: \textit{primitive}, \textit{struct}, and \textit{array} type objects. \\
\end{minipage}

\begin{minipage}{0.7\dimexpr\paperwidth}
\p{\textit{Primitive type objects}}\hspace{-1em} are type objects without any internal structure. All 
primitive type objects are predefined in the system. 

\begin{lstfloat}
\begin{lstlisting}[style=JS,caption=Primitive type objects]
var any;                     //  any (uninitialized)

var u8  = 254;               //  8-bit unsigned integers 
var u16 = 65534;             // 16-bit unsigned integers 
var u32 = 4294967294;        // 32-bit unsigned integers 
var u64 = (2**64 - 1);       // 64-bit unsigned integers 


var i8  = -127;              //  8-bit signed integer 
var i16 = -32767;            // 16-bit signed integer
var i32 = -2147483647;       // 32-bit signed integer 
var i64 = -2**63;            // 64-bit signed integer 

var f32 = 1.123456;          // 32-bit floating point
var f64 = 1.123456789012345; // 64-bit floating point

var str = "";                // String primitive
var obj = Object             // Object primitive

\end{lstlisting}
\end{lstfloat}
%% The majority of the primitive types are simple scalar types, but
%% they also include three reference types (any, object, and string).
%% The reference types are considered opaque, which means that users
%% cannot gain access to a raw array buffer containing instances of
%% these types.
%%%% page break in minipage %%%%%
\end{minipage}

%%%% page break in minipage %%%%%
\begin{minipage}{0.7\dimexpr\paperwidth}
\p{\textit{Struct type objects}}\hspace{-1em} can be composed as structures using the StructType constructor:

\begin{lstfloat}
\begin{lstlisting}[style=JS,caption=Struct type objects]
var Point = new StructType({ x:int8, y:int8 });
var Line  = new StructType({ from:Point, to:Point };
\end{lstlisting}
\vspace{-2em}
\end{lstfloat}
\textit{Listing 6} constructs two new type objects called \textit{Point} and \textit{Line}, a structure with 
two $8$-bit integer fields, $x$ and $y$, and a structure containing two aforementioned \textit{Point} structures.
The size of each \textit{Point} will be two $(2)$ bytes, while each \textit{Line} will be four $(4)$ bytes in total;
memory is laid out continuously. Hence, structures can embed other structures, e.g., \textit{prototypes}.
%% In general, the StructType constructor takes a single object
%% as argument. For each property f in this object, there will be a
%% corresponding field f in the resulting struct type. The type of this
%% corresponding field is taken from the value of the property f, which
%% must be a type object.
\end{minipage}
%%%% page break in minipage %%%%%

\begin{minipage}{0.7\dimexpr\paperwidth}
\p{\textit{Array type objects}}\hspace{-1em} are constructed by invoking the arrayType method 
on the type object representing the array elements:

\begin{lstfloat}
\begin{lstlisting}[style=JS,caption=Array type objects]
var Points = Point.arrayType(2);
var Line2  = new StructType({ points:Points });
var Plane  = Line.arrayType(768).arrayType(1024);
\end{lstlisting}
\vspace{-2em}
\end{lstfloat}
In \textit{Listing 7}, the type \textit{Points} is defined as a two-element array of \textit{Point} structures. Array types are 
themselves normal type objects, and hence they can be embedded in structures. 
The array type \textit{Points} is then used to create the structure type \textit{Line2}. \textit{Line2}
is equivalent, in layout, to that of the \textit{Line} type; however, it is defined using a two-element array
in lieu of two distinct fields. The \textit{Plane} type creates a multideminsional array by invoking
the constructor multiple times, resulting in a $1024x768$ matrix of \textit{Points}. \\
\end{minipage}
\end{center}
% It's because of \textit{Objects}; JavaScript is a prototype-inheriting language with six (6) atomics and one (1) structural type -- the \textit{Object}. Arrays, Functions, Classes, Modules, Loops, State-machines, Closures... all are simply derivatives of the parent \textit{JSObject} structural prototype. 

Reiterating our state-of-affairs: We have the entire [dynamic, weakly-typed, prototyped-based] language as our disposal; 
syntactic correctness difficult to determine, semantic equivalence\footnotemark
\footnotetext{Optimizations of this problem are exactly what JIT compilers were designed for.}
even more so. The standard slew of problems are faced; insufficient CFG/DFG mutation controls, insufficient 
targeting leading to resource waste, [...]. The \textsc{ECMA-262} language space is simply far too large. 

To improve semantic control, an AST can be used; to improve speed, bytecode can be used; A custom intermediate
representation could even be written. The aforementioned techniques aim to reduce the degrees-of-freedom of the, semantically sound and
syntacticaly correct, input space. Narrowing the constraints on the input space helps to reduce complexity and improve target accuracy; 
we posit that we can recieve the same benifit while maintaining degrees-of-freedom.

%---------------------------------------------------------------
\subsection{JavaScript Semantics: A Breakdown}

\subsubsection{What is \textit{Resolution}?}

\textbf{\textit{Resolution}} is defined here ala \textit{Display Resolution}. In other words, as degrees-of-freedom vary, so too 
does our observable. In QFT, we know this term as \textit{dimensionality} or \textit{cardinality}; we use the former
and latter interchangeably.

Classical representations of programming languages aim to reduce dimensionality in an effort to ease lexical, syntactical, and 
semantical analysis. This result is a \textit{discrete resolution} of a provided input sample program, $P$. Examples of the 
aforementioned discrete resolutions can be see in \textit{Appendix A: Cononical Representations}.

\subsubsection{A Modest Proposition}
Given a vector representation of a sample program, $P_{\lvert\psi\rangle}$, renormaliztion group transformations of 
$\lvert P \rangle$ will validate universality for cononical representations, $P_{\gamma}$, $P_{\rho}$, $P_{\tau}$. As a 
result, dimensionality can be preserved while maintaining a discrete resolution, i.e., logically equivalent semantical 
mutations exist for all semantic mutations at all resolutions of $P$.

\p{Challanges}
Assigning a semantic label to a code snippet (such as a name to a method) is an example for a class of problems that require 
a compact semantic descriptor of a snippet. The question is how to represent code snippets in a way that captures some semantic 
information, is reusable across programs, and can be used to predict properties such as a label for the snippet. This leads 
to three challenges:
%
\begin{enumerate}
  \item Representing a snippet in a way that enables RG transformations
  \item Learning which parts in the representation are relevant to prediction of the desired property
  \item Learning the order of importance (precedence) of the part
\end{enumerate}

%---------------------------------------------------------------
\subsection{Degrees of Freedom and Grammar-based Fuzzing}
%% Universiality %% 
%% we must identify non-negligible perturbative and universal semantic forms withoutimpeding on the language's degrees of freedom.



%---------------------------------------------------------------
\subsection{Chalk Trait SMT}

%---------------------------------------------------------------
%\subsection{Swift v. Rust}
%===============================================================
